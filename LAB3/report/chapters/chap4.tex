%%%%%%%%%%%%%%%%%%%%%%%%%%%%%%%%%%%%%%%%%%%%%%%%%%%%
% This will help you in writing your homebook
% Remember that the character % is a comment in latex
%
% chapter 4
\chapter{Hazards}
\label{cha4}

\section{Data Hazards Unit}

The data Hazard Unit detects the need of a value that has not been read from 
data memory yet. This problem is detected in the Decode stage, when one of the 
two source register is the destination of a load instruction in the previous cycle. 
To avoid this problem a stall is inserted in the pipeline to wait for the load of the data 
from memory. 

To insert the stall:
\begin{itemize}
    \item The program counter is not updated;
    \item The IF/ID pipeline is not updated and keeps storing the previous instruction;
    \item All control signals are set to zero, regardless of the Control Unit, using a multiplexer.
\end{itemize}
The stall lasts only one clock cycle.

After reading the value from memory the forwarding unit will place it in the proper 
position in Execution stage.

\section{Branch Detection}

The branches are detected in Decode stage, where a dedicated adder will calculate always the new value where to jump.
Two situations occur:
\begin{itemize}
    \item unconditional branch, the jump is always taken and PC is always upgrade with the offset value;
    \item conditional branch, before proceeding with jump a condition must be verified, for this Laboratory
     exists only the possibility of the equivalence of two content registers. It takes care also about the fact that
     some register value could be forwarded from other stages. 
\end{itemize}

If a branch is not taken nothing happens and the following instruction will load like usual. 
In case of branch taken the instruction before is flushed and substitute with a NOP, 
in the next clock cycle, the instruction loaded will be the 
one reached through the new offset program counter, calculated with the previous jump instruction.

