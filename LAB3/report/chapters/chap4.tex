%%%%%%%%%%%%%%%%%%%%%%%%%%%%%%%%%%%%%%%%%%%%%%%%%%%%
% This will help you in writing your homebook
% Remember that the character % is a comment in latex
%
% chapter 4
\chapter{Hazards}
\label{cha4}

\section{Data Hazards Unit}

The data Hazard Unit detect the needed of a value that has not been read from 
data memory yet. This problem is detected in the Decode stage, when one of the 
two source register is the destination of a load instruction in the previous one. 
To avoid this problem a stall is insert in the pipeline to waiting for load the data 
from memory. 

To insert the stall:
\begin{itemize}
    \item The program counter is not update;
    \item The IF/ID pipeline is not update and keep storing the previous instruction;
    \item All control signal are set to zero, regardless of the Control Unit, using a multiplexer.
\end{itemize}
The stall lest only one clock cycle.

After reading the value from memory the forwarding unit will place it in the proper 
position in Execution stage.

\section{Branch Detection}

The branchs are Detected in Decode stage, where a dedicated adder will calclated always the new value where jump.
Two situations exist:
\begin{itemize}
    \item uncoditional branch, the jump is always taken and PC is always upgrade with the offset value;
    \item confitonal branch, before to procede with jump a condition must be verified, for this Laboratory
     exists only the possibility of the equivalence of two content registers. It takes care also about the fact that
     some register value could be forwarding from other stages. 
\end{itemize}

If a branch is not take nothing happen and the following instructioni will load like usual. 
In case of branch taken the instruction before is flushed and sobstitute with a NOP, 
in the next clock cycle, the instruction load will be the 
one reached through the new offset program cunter, calculated with the previous jump instruction.

\section{Absolute Module}

An entity has been written in order to allow the computation of absolute value of an integer register.
This madule has two input ports and one output port. The value on the first input port is the one to convert 
through the absolute function. To achive this result it sees the last bit of the signal if a zero is detected 
no operations are perfomed, while if a 1 is detected alle the signals are complemented and is added a one. 
These are the steps to covert a number, in this case from a negative to a positve value, according absolute function.
At the end the second input is summed to the convert one and the result is sent to the output.
This component has been inserted into the ALU to increase its operational capacity.