%%%%%%%%%%%%%%%%%%%%%%%%%%%%%%%%%%%%%%%%%%%%%%%%%%%%
% This will help you in writing your homebook
% Remember that the character % is a comment in latex
%
% chapter 1
\chapter{Reference model development}
\label{chap1}

%%%%%%%%%%%%%%%%%%%%%%%%%%%%%%%%%%%%%%%%%%%%%%%%%%%%%%%%%%%
% you can organize a chapter using sections -> \section{Simulating an inverter}
% or subsections -> \subsection{simulating a particular type of inverter}

%%%%%%   First section
\section{Introduction}

The goal of this laboratory is to design a Finite Impulse Filter filter (FIR) with a cut frequency of 2 kHz, and then applying some optimization 
techniques such as unfolding and pipelining to the basic structure.
Filter has is design according two parameter: order and number of bits. The order employ for the following filter
is 10 and the number of bits are 9.

A prototype version of the filter has been developed in C language and Matlab in order to be able to compare them with the
results coming from the simulation of the HDL design.

\section{Design the filter with Matlab}

First step is the generation of coefficients. To do this Matlab function fir1 has been used.
The coefficients are shown in table \ref{tab:1}. % here is the reference to the table below

\begin{table}[ht]
\centering
\begin{tabular}{c|c|c}
\toprule
Number & Quantize & Normalize \\
\midrule
0 & -1 & 1 \\
1 & -2 & 1 \\
2 & -4 & 1 \\
3 & 8 & 0 \\
4 & 35 & 1 \\
5 & 50 & 1 \\
6 & 35 & 1 \\
7 & 8 & 1 \\
8 & -4 & 1 \\
9 & -2 & 1 \\
10 & -1 & 1 \\
\bottomrule
\end{tabular}
\caption{All coefficients.}
\label{tab:1}
\end{table}

At this point, another Matlab script is executed in order to perform
different simulatios with prototype filter with a cut-off frequency of 2 kHz and a sampling frequency of 10 kHz, taking as 
input signal the average value between two sinusoidal waves of frequency of 500 HZ and 4.5 kHz respectively.
After this execution two files have been generated:

\begin{enumerate}
	\item \emph{sample.txt}, which contains the sample values that have fed the input of FIR;
	\item \emph{result.txt}, which contains the output values that has been elaborated from our FIR.
\end{enumerate}

\section{C prototype}

The C-language script simualtes a FIR filter implementing the following  relation:

\begin{displaymath}
y_i = \sum_{n=0}^{10}{x_{i-n} \cdot b_n}
\end{displaymath}

Thanks to this script is possible to compare the performance of the fixed-point version with respect to 
the Matlab computation results.

\subsection{Evaluate the THD}

The purpose of this step is to evaluate the Total Harmonic Distortion (THD), trying to obtain a maximum 
value of -30dB. 
If THD exceeds the maximum tolerated value, it is necessary to increase 
the bit numbers in order to reduce its amount, while if there is a gap between the maximum tolerated value 
and the obtained one, it is possible to reduce bit numbers and thus the complexity of the FIR implementation.

%% add value of THD caculated each time

With 9 bits used for data, obtained THD is -40 dB. 
Reducing the number of bits to 8 the obtained value of THD  
is -33 dB, thus still acceptable. 
Applying a further bit-number reduction, thus with a 7-bit parallelism, the value of THD exceeds the maximum allowed
amount, reaching the value of -27dB.

At the end of this analysis, it has been decided to use 8 bits for the final implementation of the  FIR, 
in order to reduce the area while still accomplish maximum THD amount.

%% vedi i ths a 9, 8 e 7 bits