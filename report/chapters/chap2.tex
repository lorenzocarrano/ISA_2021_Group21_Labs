%%%%%%%%%%%%%%%%%%%%%%%%%%%%%%%%%%%%%%%%%%%%%%%%%%%%
% This will help you in writing your homebook
% Remember that the character % is a comment in latex
%
% chapter 2
\chapter{VLSI implementation}
\label{cha2}
\section{Starting architecture development}

The purpose of this section is to develop in VHDL the architecture of the previously designed filter. 
The architecture of the filter is composed by four elements:

\begin{itemize}
    \item Adders
    \item Multipliers
    \item Flipflops
    \item Registers
\end{itemize}

The 8-bit input is recived and then propagated through a chain of 10 registers;
the output of each register is multiplied by the corresponding coefficient, and the results are summed together to form the
filter's output. 
All registers use VIN as an enable signal, in order to avoid unwanted propagation of data. The VIN signal, delayed of two clock cycles, is also used to 
drive VOUT. Every input and output signal is loaded or produced by registers or flipflops, to reduce the risk of interference from external signals.

%%Image of FIR filter

\section{Simulation}

The design was simulated using a testbench written in both Verilog and VHDL. The testbench is composed of four disinct entities:

\begin{itemize}
    \item \textbf{clk\_gen:} generates a clock signal of the specified frequency, and a reset signal.
    \item \textbf{data\_maker:} reads the samples.txt file and provides an input every clock cycle and its validity using the VIN signal. 
    \item \textbf{data\_sink:} recives the outputs of the filter every clock cycle and writes them in the output.txt file if VOUT is equal to 1. 
    \item \textbf{tb\_fir:} is the testbench top entity written in Verilog.
\end{itemize}

At the end of the simulation