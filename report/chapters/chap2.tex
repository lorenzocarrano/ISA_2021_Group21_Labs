%%%%%%%%%%%%%%%%%%%%%%%%%%%%%%%%%%%%%%%%%%%%%%%%%%%%
% This will help you in writing your homebook
% Remember that the character % is a comment in latex
%
% chapter 2
\chapter{VLSI implementation}
\label{cha2}
\section{Starting architecture development}

The purpose of this section is to develop in VHDL the architecture of the previously designed filter. 
The architecture of the filter is composed by four elements:

\begin{itemize}
    \item Adders
    \item Multipliers
    \item Flipflops
    \item Registers
\end{itemize}

The 8-bit input is recived and then propagated through a chain of 10 registers;
the output of each register is multiplied by the corresponding coefficient, and the results are summed together to form the
filter's output. 
All registers use VIN as an enable signal, in order to avoid unwanted propagation of data. The VIN signal, delayed of two clock cycles, is also used to 
drive VOUT. Every input and output signal is loaded or produced by registers or flipflops, to reduce the risk of interference from external signals.

%%Image of FIR filter

\section{Simulation}

The design was simulated using a testbench written in both Verilog and VHDL. The testbench is composed of four disinct entities:

\begin{itemize}
    \item \textbf{clk\_gen:} generates a clock signal of the specified frequency, and a reset signal.
    \item \textbf{data\_maker:} reads the samples.txt file and provides an input every clock cycle and its validity using the VIN signal. 
    \item \textbf{data\_sink:} recives the outputs of the filter every clock cycle and writes them in the output.txt file if VOUT is equal to 1. 
    \item \textbf{tb\_fir:} is the testbench top entity written in Verilog.
\end{itemize}

%%screen of waveforms

At the end of the simulation the values stored in Output.txt were compared with the ones produced by the C prototype. The two files are equal, which
means that the filter is behaving correctly. 

%%table with first results of files

\section{Logic synthesis}

After the simulation the design must be synthetized. To estimate the maximum working clock frequency of the filter,
the clock period in the design compiler is set to 0 ns. In this way the compiler optimizes the circuit as much as possible,
and the negative slack of the timing report corresponds to the maximum clock frequency. 

After running the synthesis at the computed frequency the area is evaluated.

%%table with frequency & area

It is requested to set the frequency to 25\% of the maximum value. After the synthesis a new area estimation is produced.

%%new area

The constraints on the clock have a significant role in the estimation of the area: by allowing the frequency to be lower, 
the size of the circuit will be smaller.

The Design Compiler produces the Verilog netlist of the synthetized circuit and a .sdf file containing the circuit's delays.
Those files are used by Modelsim to generate 