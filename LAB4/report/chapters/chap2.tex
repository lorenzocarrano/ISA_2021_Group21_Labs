%%%%%%%%%%%%%%%%%%%%%%%%%%%%%%%%%%%%%%%%%%%%%%%%%%%%
% This will help you in writing your homebook
% Remember that the character % is a comment in latex
%
% chapter 2
 
\chapter{Adder}
\label{cha2}

The first step has been to test the correctness of the behavior of the System Verilog description of
an adder by using QuestaSim.
In order to evaluate the performance, in terms of coherence with the expected behavior, it is needed
to check the amount of matches and mismatches.
For each input sequence, it was observed that the output of the adder description to be tested
and the output of the gold model were coincidental, thus we can conclude that the overall 
behavior of the described adder is correct.

\section{Changing the parallelism}
Changing the parallelism means to modify both the interface of the unit under test and
size of each input/output sequence.
Even in this case, no mismatches have been detected by comparison with the golden model.

\section{Constraints Definition}
It is possible to define some constraints over the randomly generated input sequences 
employed in the test, for example using the inside operator, to define a range of values
for one of the inputs of the adder.
Even in this case, no mismatches have been observed.

\section{Incorrect Gold Model}
At this point, a golden model with an incorrect behavior has been employed for the test, thus
in this case it is expected to observe mismatches inside each record of the transcript.
For each mismatch, a warning is signaled in the transcript, and an error is reported inside
the final summary.

\section{Forcing Overflow Condition}
At this step the behavior of the unit under test in case of overflow has been tested.
In order to do that, it has been necessary to set proper constraints over the input sequences.
It has been possible to observe that the adder correctly handles overflow conditions.