%%%%%%%%%%%%%%%%%%%%%%%%%%%%%%%%%%%%%%%%%%%%%%%%%%%%
% This will help you in writing your homebook
% Remember that the character % is a comment in latex
%
% chapter 1
\chapter{Introduction}
\label{chap1}

During the verificatoin steps of HDL design different testbench had been used to 
check the correctness of implementation. Each of them has been manualy modified to 
verify the correct bheavior of design under different input oand no automatic tools 
had been used in order to automatize this process.

The UVM is the lastest tecniquie used to create modular testbench. It's a free framework 
based on System Verilog and exploid the Object-Oriented paradigma. This paradigm entails
the separation of concerns while building the testbench architecture and the specific test that
can be performed: multiple tests can be applied to the same architecture just by modifying the
input stimuli sequence (and constraints) and by selecting different Design Under Test (DUT)
properties to be observed. Alternatively, same tests can be applied to different testbench architectures.

In this laboratory the generation of different input has been randomly and the right output to compare has 
been elaborated based on a reference model, implemented through the UVM class, properly create according 
the DUT. In this way the only operation to perfome in order to change the testbench are:
\begin{itemize}
    \item choose the proper DUT
    \item select a range for input value
    \item modify the model reference and the FSM to follow the design bheavior
\end{itemize}


            