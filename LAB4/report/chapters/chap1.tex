%%%%%%%%%%%%%%%%%%%%%%%%%%%%%%%%%%%%%%%%%%%%%%%%%%%%
% This will help you in writing your homebook
% Remember that the character % is a comment in latex
%
% chapter 1
\chapter{Introduction}
\label{chap1}

During the verification steps of HDL design different testbenches have been used to 
check the correctness of implementation. Each of them has been manually modified to 
verify the correct behavior of the design under different inputs, and no automatic tools 
had been used in order to automatize this process.

The UVM is the latest technique used to create modular testbenches. It's a free framework 
based on System Verilog and exploits the Object-Oriented paradigm. This paradigm entails
the separation of concerns while building the testbench architecture and the specific test that
can be performed: multiple tests can be applied to the same architecture just by modifying the
input stimuli sequence (and constraints) and by selecting different Design Under Test (DUT)
properties to be observed. Alternatively, the same tests can be applied to different testbench architectures.

In this laboratory the generation of different input has been random and the right output to compare has 
been elaborated based on a reference model, implemented through the UVM class, properly created according 
to the DUT. In this way the only operation to performed in order to change the testbench are:
\begin{itemize}
    \item choose the proper DUT;
    \item select a range for input values;
    \item modify the model reference and the FSM to follow the design behavior.
\end{itemize}


            