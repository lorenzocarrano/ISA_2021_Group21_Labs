%%%%%%%%%%%%%%%%%%%%%%%%%%%%%%%%%%%%%%%%%%%%%%%%%%%%%%%%%%%%%%%%%%%%%%%%%
%
% This file defines the style for your homebook
% You don't need to edit it any more, if not to 
% change the authors name:
%
% Search below for the keyword:   GROUP
% insert your group number
%
% Search below for the keyword:   AUTHORS
% insert the name of the authors
%
%%%%%%
% Now to update the dexcription  of your work you will 
% use the file ``master.tex'' in the current directory
% following the instructions in it 
%
%%%%%% 
%%%%%%  
%%%%%%
% If you want to compile your document you have TWO ways
% depending on the fact that 
% 	1) you have inserted only postcript images in your tex file 
%		---> then go to MODE 1
%	2) you have inserted other kind of images (jpg..) in your tec file
%		---> then go to MODE 2
%
% MODE 1 
%simple type:
% 	latex homebook.tex
%
% If the compilation runs succesfully and you want to see the results type:
% 	xdvi homebook.dvi &
% and use the menus to go through the document
%
% If you want to create a pdf type:
% 	dvipdfm homebook.dvi
%
% a homebook.pdf file is created
% you can see it using the command:
% 	acroread homebook.pdf &
%
%
% MODE 2
% simple type:
%	pdflatex homebook.tex
%
% If the compilation runs succesfully you directly have the pdf file
% and you can see it using the command:
%       acroread homebook.pdf &
%
% 
%%%%%%%%%%%%%%%%%%%%%%%%%%%%%%%%%%%%%%%%%%%%%%%%%%%%%%%%%%%%%%%%%%%%%%%%%%
\documentclass[11pt,  english, makeidx, a4paper, titlepage, oneside]{book}
\usepackage{babel}
\usepackage{fancyhdr}
\usepackage{makeidx}
\usepackage{titlesec}
\usepackage{listings} 
\usepackage{booktabs}
\usepackage{multirow}


\newenvironment{listato}{\footnotesize}
                        {\normalsize }


%\pagestyle{empty}

\textwidth 15.5cm
\textheight 23cm
\topmargin -1cm
\oddsidemargin -0.0cm
\linespread{1.1}

\pagestyle{fancy}
\lhead{}
\chead{Integrated Systems Architecture}
\lfoot{}
\cfoot{}
\rfoot{}
\rhead{\thepage}

\usepackage{graphicx}
\usepackage{amsmath}
\usepackage{amsfonts}
\usepackage{amsthm}
\usepackage{amssymb}
\usepackage{tabularx}
\usepackage[mediumspace,mediumqspace,Grey,squaren]{SIunits}
%\oddsidemargin -1.1cm

\titleformat{\chapter}[display]
{\normalfont\Large\filcenter\sffamily}
{\titlerule[0.5pt]%
\vspace{1pt}
\titlerule
\vspace{1pc}
\LARGE\MakeUppercase{\chaptertitlename} \thechapter
}
{1pc}
{\titlerule
\vspace{1pc}
\Huge}

\newcommand{\SubSubSection}[1]{\subsubsection{\bf Exercise   ~#1}}

\newcommand{\homework}[1]{\subsubsection{\bf Homework   ~#1}}

\newcommand{\Solution}{\subsubsection{\bf Solution}}




\makeindex
\begin{document}
\frontmatter
\begin{titlepage}
\vspace{2cm}
\centerline{
\includegraphics[width=3cm]{./logopoli_new.png}} 
\vspace{0.5cm}
\centerline{\LARGE Politecnico di Torino}
\bigskip
\centerline{\Large III Facolt\`a di Ingegneria}
\vspace{3.5cm}
\centerline{\Huge\sf Laboratory 1}
\bigskip
\centerline{\Huge\bfseries\sf Design And Implementation Of A Digital Filter}
\vspace{2cm}
\centerline{\LARGE Master degree in Electrical Engineering}
\vspace{4cm}
%%%%%%%%%%%%%%%%%%%%%%%%%%%%%%%%%%%%%%%%%%%%%%%%%%%%%%%
% GROUP
% Change the name of your group below
%
\centerline{\Large Authors: Group 21}
\vspace{2cm}
%
%%%%%%%%%%%%%%%%%%%%%%%%%%%%%%%%%%%%%%%%%%%%%%%%%%%%%%%
% AUTHORS
% Change the name of the Group participants here
%
\centerline{Dilillo Nicola S284963}
\centerline{Moncalvo Stefano S290315}
\centerline{Carrano Lorenzo S281565}
%
%%%%%%%%%%%%%%%%%%%%%%%%%%%%%%%%%%%%%%%%%%%%%%%%%%%%%%
\vspace{1cm}
\centerline{\today}
\vspace{1cm}
%{\scriptsize Many thanks to Prof. Mariagrazia Graziano for providing us with this template.}
\end{titlepage}

\tableofcontents

%%%%%%%%%%%%%%%%%%%%%%%%%%%
% 
\mainmatter
\lstset{language=VHDL}

%%%%%%%%%%%%%%%%%%%%%%%%%%%%%%%%%%%%%%%%%%%%%%%%%%%%%%
%    
% HERE IS WHERE YOU INCLUDE YOUR CHAPTERS
%
%%%%%%%%%%%%%%%%%%%%%%%%%%%%%%%%%%%%%%%%%%%%%%%%%%%%
% This will help you in writing your homebook
% Remember that the character % is a comment in latex
%
% chapter 1
\chapter{Introduction}
\label{chap1}

During the verification steps of HDL design different testbenches have been used to 
check the correctness of implementation. Each of them has been manually modified to 
verify the correct behavior of the design under different inputs, and no automatic tools 
had been used in order to automatize this process.

The UVM is the latest technique used to create modular testbenches. It's a free framework 
based on System Verilog and exploits the Object-Oriented paradigm. This paradigm entails
the separation of concerns while building the testbench architecture and the specific test that
can be performed: multiple tests can be applied to the same architecture just by modifying the
input stimuli sequence (and constraints) and by selecting different Design Under Test (DUT)
properties to be observed. Alternatively, the same tests can be applied to different testbench architectures.

In this laboratory the generation of different input has been random and the right output to compare has 
been elaborated based on a reference model, implemented through the UVM class, properly created according 
to the DUT. In this way the only operation to performed in order to change the testbench are:
\begin{itemize}
    \item choose the proper DUT;
    \item select a range for input values;
    \item modify the model reference and the FSM to follow the design behavior.
\end{itemize}


            
%%%%%%%%%%%%%%%%%%%%%%%%%%%%%%%%%%%%%%%%%%%%%%%%%%%%
% This will help you in writing your homebook
% Remember that the character % is a comment in latex
%
% chapter 2
\chapter{VLSI implementation}
\label{cha2}
\section{Starting architecture development}

The purpose of this section is to develop in VHDL the architecture of the previously designed filter. 
The architecture of the filter is composed by four elements:

\begin{itemize}
    \item Adders
    \item Multipliers
    \item Flipflops
    \item Registers
\end{itemize}

The 8-bit input is recived and then propagated through a chain of 10 registers;
the output of each register is multiplied by the corresponding coefficient, and the results are summed together to form the
filter's output. 
All registers use VIN as an enable signal, in order to avoid unwanted propagation of data. The VIN signal, delayed of two clock cycles, is also used to 
drive VOUT. Every input and output signal is loaded or produced by registers or flipflops, to reduce the risk of interference from external signals.

%%Image of FIR filter

\section{Simulation}

The design was simulated using a testbench written in both Verilog and VHDL. The testbench is composed of four disinct entities:

\begin{itemize}
    \item \textbf{clk\_gen:} generates a clock signal of the specified frequency, and a reset signal.
    \item \textbf{data\_maker:} reads the samples.txt file and provides an input every clock cycle and its validity using the VIN signal. 
    \item \textbf{data\_sink:} recives the outputs of the filter every clock cycle and writes them in the output.txt file if VOUT is equal to 1. 
    \item \textbf{tb\_fir:} is the testbench top entity written in Verilog.
\end{itemize}

%%screen of waveforms

At the end of the simulation the values stored in Output.txt were compared with the ones produced by the C prototype. The two files are equal, which
means that the filter is behaving correctly. 

%%table with first results of files

\section{Logic synthesis}

After the simulation the design must be synthetized. To estimate the maximum working clock frequency of the filter,
the clock period in the design compiler is set to 0 ns. In this way the compiler optimizes the circuit as much as possible,
and the negative slack of the timing report corresponds to the maximum clock frequency. 

After running the synthesis at the computed frequency the area is evaluated.

%%table with frequency & area

It is requested to set the frequency to 25\% of the maximum value. After the synthesis a new area estimation is produced.

%%new area

The constraints on the clock have a significant role in the estimation of the area: by allowing the frequency to be lower, 
the size of the circuit will be smaller.

The Design Compiler produces the Verilog netlist of the synthetized circuit and a .sdf file containing the circuit's delays.
Those files are used by Modelsim to generate 
%%%%%%%%%%%%%%%%%%%%%%%%%%%%%%%%%%%%%%%%%%%%%%%%%%%%
% This will help you in writing your homebook
% Remember that the character % is a comment in latex
%
% chapter 3
\chapter{All Multiplier}
\label{cha3}


% \input{./chapters/chap_name}
% and so on
%
%%%%%%%%%%%%%%%%%%%%%%%%%%%%%%%%%%%%%%%%%%%%%%%%%%%%%%
%    
% HERE IS WHERE YOU INCLUDE YOUR APPENDICES (IF ANY)
%
%\appendix
%\input{./appendices/appendix1}
%\input{./appendices/appendix2}
% and so on
%
%%%%%%%%%%%%%%%%%%%%%%%%%%%%%%%%%%%%%%%%%%%%%%%%%%%%%%
 
\end{document}